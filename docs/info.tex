\documentclass[12pt]{article}
\usepackage[
  a4paper,
  margin=1in,
  headsep=4pt, % separation between header rule and text
]{geometry}
\usepackage{xcolor}
\usepackage{fancyhdr}
\usepackage{tgschola}
\usepackage{lastpage}
\usepackage{graphicx}
\usepackage[export]{adjustbox}[2011/08/13]
\usepackage[natbibapa]{apacite}
\usepackage{subcaption}
\usepackage{hyperref}
\usepackage{amsmath}

\newcommand{\blue}[1]{\textcolor{blue}{#1}}
\newcommand{\red}[1]{\textcolor{red}{#1}}

\newcommand{\wds}{\widetilde{\Delta\Sigma}}
\newcommand{\ds}{\Delta\Sigma}
\newcommand{\model}{\vec{\theta}}

\begin{document}

%%%%%%%%%%%%%%%%%%%%%%%%
\noindent
{\bf Mass bias basics}

In galaxy cluster analysis, the mass bias is defined as the difference between a cluster's true mass $M_{\rm true}$ and its observed (i.e. measured) mass $M_{\rm obs}$, written as $\Delta M = M_{\rm obs}-M_{\rm true}$. Sometimes this is written as a ratio: $\mathcal{C} = M_{\rm true}/M_{\rm obs} = 1-\Delta M/M_{\rm obs}$.

When measuring galaxy cluster masses, a mass bias is caused by systematics, sometimes in combination with each other. Specifically, the mass bias is caused by systematics that are unaccounted for which affects observations of the cluster. Let's take stacked weak gravitational lensing as an example. For stacked lensing, the mass $M$ we are measuring is the average mass of the stack.

%%%%%%%%%%%%%%%%%%%%%%%%
\vspace{12pt}
\noindent
{\bf Weak lensing}

The weak lensing profile for a stack with mean mass $M$ is defined as 
%
\begin{equation}
	\label{eq:deltasigma_def}
	\Delta\Sigma(R|M) = \bar{\Sigma}(<R|M) - \Sigma(R|M)\,,
\end{equation}
%
where $\Delta\Sigma$ is the differential surface mass density at radius $R$, $\bar{\Sigma}(<R)$ is the average surface mass density below $R$, and $\Sigma(R)$ is the surface mass density at $R$. The surface mass density is given as the line-of-sight integral of the overdensity of the halo a distance $R$ from its center:
%
\begin{equation}
	\label{eq:sigma_def}
	\Sigma(R|M) = \int_{-\infty}^{\infty} {\rm d}r_z\ \Delta\rho\left(\sqrt{R^2+r_z^2}|M\right) = 2\bar{\rho}\int_0^\infty \xi_{\rm hm}\left(\sqrt{R^2+r_z^2}|M\right)\,,
\end{equation}
%
where $\Delta\rho(r|M) = \rho(r|M) - \bar{\rho} = \bar{\rho}\xi_{\rm hm}$ is the difference between the halo density and the background density ($\rho_{\rm crit}$ or $\rho_{m}$, depending on halo definition) and $\xi_{\rm hm}$ is the halo-matter correlation function. By measuring the shape of galaxies behind galaxy clusters, we can get an estimate of the lensing profile $\wds$. In other words, $\wds$ would be our ``data'', and the parameters that go into $\ds$ as our ``model''.

In measuring the mass of a cluster stack, we want to measure the posterior $P(\vec{\theta}|\wds)$, or the distribution of our model parameters $\vec{\theta}$ given our data. Using Bayes' theorem we can write this in terms of the likelihood of our data and a prior
%
\begin{equation}
	\label{eq:posterior}
	P(\vec{\theta}|\wds) = \frac{P(\wds|\vec{\theta})P(\vec{\theta})}{P(\wds)}\,,
\end{equation}
%
where we can neglect the prior on the data, since it is just a normalization, and $M\in\model$. A mass bias occurs when computing $\ds$ from the model $\model$ results in a biased estimate of the lensing signal. In other words, $\Delta\Sigma(M+\Delta M) \approx \wds(M)$. Therefore, when we evaluate the posterior of the parameters, the marginalized posterior on $M$ yields an expectation value of $M+\Delta M$. If not corrected for, this mass bias propagates into cosmological parameters (see, e.g., the Planck cluster cosmology papers).

One thing to note is that the mass bias depends on the likelihood of the data $P(\wds|\vec{\theta})$. Oftentimes, the log of this likelihood looks like
%
\begin{equation}
	\label{eq:likelihood}
	\ln P(\wds|\vec{\theta}) = -\frac{1}{2}\left[ \ln|C| +\sum_{i,j} (\wds - \Delta\Sigma(\model))_i C_{i,j}^{-1}(\wds - \Delta\Sigma(\model))_j  \right]
\end{equation}
%
where $C$ is the covariance matrix, and $\ds(\model)$ is our model for the lensing profile computed from our parameters. The indices $i,j$ run over radial bins $R$. Therefore, not only is our mass bias going to depend on $\ds(\model)$, but also on our covariance matrix $C$.

%%%%%%%%%%%%%%%%%%%%%%%%
\vspace{12pt}
\noindent
{\bf Measuring mass bias}

Using simulations, we can estimate the mass bias and correct for it. An $N$-body simulation evolves the positions of dark matter particles from high redshift to present day. Those particles interact gravitationally, and can virialize to form bound dark matter halos, which are known to host galaxy clusters. So, in a simulation, we have to sets of data, positions of dark matter particles $\vec{r}_{\rm dm}$ and halo catalogs that consist of positions and masses of the halos $\{\vec{r}_{\rm h}, M\}_i, i\in[1,...,N_{\rm h}]$.

Galaxy clusters are identified not by their mass, but by some observable quantity, such as richness $\lambda$. Assuming we have a decent model for $P(\lambda|M)$, we can draw from this distribution and assign each halo a richness $\lambda_i$. Now, by grouping halos in bins of observable, we can make simulated stacks that have the same properties as the real data.

With our dark matter particle positions and halo catalog, one thing we can measure is the halo-matter correlation function $\xi_{\rm hm}$. This is found by pair counting, and can be computed using the following estimator:
%
\begin{equation}
	\label{eq:xihm_estimator}
	\xi_{\rm}(r) = \frac{D_hD_{dm}(r)}{RR(r)} - 1\,.
\end{equation}
%
In this equation, $D_hD_{dm}(r)$ is the number of halo-particle pairs at a distance $r = ||\vec{r}_{\rm h} - \vec{r}_{\rm dm}|| $. $RR(r)$ is the number of pairs of random points separated by $r$. This equation assumes that the number densities have all be scaled correctly, but these are all gritty details. Pair counting is usually done by a specialized code, such as \href{https://github.com/manodeep/Corrfunc}{Corrfunc} or \href{https://github.com/rmjarvis/TreeCorr}{TreeCorr}. The details of the implementations of those codes is beyond the scope of this summary, and we can assume that the $\xi_{\rm hm}$ we get out is correct.

At this point, we have halo-matter correlation functions for stacks of halos binned by some quantity in the simulation. We can convert this into a simulated $\hat{\Delta\Sigma}$ profile following \autoref{eq:deltasigma_def} and \autoref{eq:sigma_def}. That is, we integrate the correlation function along the line-of-sight to compute $\Sigma(R)$, from which we can compute $\hat{\ds}$. There are some details about how to extrapolate $\xi_{\rm hm}(r)$ to very small scales, but they are also a minor detail. For reference, in this package, the halo profile snaps to an NFW profile at those small scales, even though this profile isn't totally correct.

Finally, we plug our simulated profile $\hat{\ds}$ into \autoref{eq:likelihood} and measure the halo mass. Since we know the masses of the halos in the simulations, we know $M_{\rm true}$. The mass we recover is $M_{\rm obs}$, meaning we have an estimate of the mass bias $\mathcal{C}$. More specifically, we have a full posterior for $\mathcal{C}$, which includes a mean and variance.

%%%%%%%%%%%%%%%%%%%%%%%%
\vspace{12pt}
\noindent
{\bf Systematics}

I mentioned earlier that systematics are the cause of the mass bias. Even though we can compute a mass bias and correct for it, we should always try to minimize the bias by incorporating systematics into our simulated profile. Here is a list of systematics that can be applied in this package and how they affect the likelihood:
%
\begin{itemize}
\item Multiplicative bias of the form $A\times\ds$. A multiplicative bias $A$ is almost completely degenerate with mass. The degeneracy is only slightly broken since the profile shape has a weak mass dependence. This dependence is not likely to be detected in DES or most of LSST, meaning for now a multiplicative bias is completely degenerate with mass. This kind of a bias can occur from shape and photo-z biases.
\item Boost factors $\mathcal{B}(R)\ds$. This is also known as contamination of the source galaxy population, from galaxies either in the foreground or from within the cluster. This has the effect of re-weighting the terms in \autoref{eq:likelihood}. In this package, the functional form and free parameters of the boost factor profile can be specified or left out entirely.
\item Miscentering $(1-f_{\rm mis})\ds + f_{\rm mis}\ds_{\rm mis}$, where $f_{\rm mis}$ is the fraction of clusters in a stack that are miscentered. Miscentering is caused by the misidentification of cluster centers, which suppresses the lensing profile on small scales. This is a linear effect on the lensing profile, meaning it also has the effect of re-weighting terms in the likelihood. As was the case for boost factors, the miscentering parameters and form of the miscentering distribution can be specified or left out.
\item Model bias. This refers to the fact that the model we use in \autoref{eq:likelihood} $\ds(\model)$ is not a perfect model. For instance, halo profiles are not NFW. The halo bias $b(M)$ at large scales might not be accurately modeled. Etc. This systematic is always present in this package, and can only be mitigated by finding better lensing models. This can be accomplished through simulation based models (emulators) and/or modeling additional effects on the lensing profile (triaxiality, projection effects, subhalos, intrinsic alignments, etc.).
\end{itemize}
%

%%%%%%%%%%%%%%%%%%%%%%%%
\vspace{12pt}
\noindent
{\bf Modeling the mass bias}

Once a mass bias $\mathcal{C}$ is computed for each simulated stack, we can model the mass bias. This is a necessary step, because we can have more or less data in our simulation compared to real data, and we want to leverage the simulation as much as possible in order to apply it to the mass measurements on real data. The alternative would be to apply the calibration for the $i$th simulated stack to the mass measurement of the $i$th real stack, so $M_{i} = \mathcal{C}_i M_{\rm obs,i}$.

A sensible model for the mass bias is a power law in observable and redshift, as well as some intrinsic scatter.
%
\begin{align}
	\label{eq:mass_bias_form}
	\langle \mathcal{C}|\lambda,z\rangle  & = \mathcal{C}_0\left(\frac{\lambda}{\lambda_0} \right)^\alpha \left(\frac{1+z}{1+z_0}\right)^\beta \\
	{\rm Var}(\mathcal{C}|\lambda,z) & = \sigma_0^2
\end{align}
%
Here, the model parameters are the amplitude of the mass bias $\mathcal{C}_0$, the powers of the observable and redshift scaling $\alpha$ and $\beta$, and the intrinsic scatter $\sigma_0$.

\noindent
\red{More to come...}


\end{document}
