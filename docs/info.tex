\documentclass[12pt]{article}
\usepackage[
  a4paper,
  margin=1in,
  headsep=4pt, % separation between header rule and text
]{geometry}
\usepackage{xcolor}
\usepackage{fancyhdr}
\usepackage{tgschola}
\usepackage{lastpage}
\usepackage{graphicx}
\usepackage[export]{adjustbox}[2011/08/13]
\usepackage[natbibapa]{apacite}
\usepackage{subcaption}

\newcommand{\blue}[1]{\textcolor{blue}{#1}}
\newcommand{\red}[1]{\textcolor{red}{#1}}

\begin{document}

\noindent
{\bf Mass bias basics}

In galaxy cluster analysis, the mass bias is defined as the difference between measuring a cluster's true mass $M_{\rm true}$ and its observed (i.e. measured) mass $M_{\rm obs}$, written as $\Delta M = M_{\rm obs}-M_{\rm true}$. Sometimes this is written as a ratio: $\mathcal{C} = M_{\rm true}/M_{\rm obs} = 1+\Delta M/M_{\rm true}$.

\end{document}
